%!TEX TS-program = xelatex
%!TEX encoding = UTF-8 Unicode

\documentclass[12pt]{extarticle}
% extarticle is like article but can handle 8pt, 9pt, 10pt, 11pt, 12pt, 14pt, 17pt, and 20pt text

\def \ititle {Central Themes in Philospohy}

\def \isubtitle {Lecture 07}

\def \iauthor {Stephen A. Butterfill}
\def \iemail{s.butterfill@warwick.ac.uk}
\date{}

%for strikethrough
\usepackage[normalem]{ulem}

\input{$HOME/latex_imports/preamble_steve_handout}

%\bibpunct{}{}{,}{s}{}{,}  %use superscript TICS style bib
%remove hanging indent for TICS style bib
%TODO doesnt work
\setlength{\bibhang}{0em}
%\setlength{\bibsep}{0.5em}


%itemize bullet should be dash
\renewcommand{\labelitemi}{$-$}

\begin{document}

\begin{multicols*}{3}

\setlength\footnotesep{1em}


\bibliographystyle{newapa} %apalike

%\maketitle
%\tableofcontents




%---------------
%--- start paste


      
\def \ititle {Who Is Responsible \\ for Global Poverty?}
 
\def \isubtitle {}
 
\begin{center}
 
{\Large
 
\textbf{\ititle}
 
}
 
 
 
\iemail %
 
\end{center}
 
All unattributed quotes are from \citet{pogge:2005_world}.

\section{Background}
 
Today there are probably substantially more than 600m people living on under \$1.90 a day
at purchasing power parity (PPP) exchange rates using 2011 prices \citep{webber:2020_we}.

 
‘The common assumption [...] is that reducing severe poverty  abroad 
at the expense of our own affluence 
 would be generous on our part, 
not something we owe, and that our failure to do this is thus at most 
a lack of  generosity that does not make us morally responsible 
 for the continued  deprivation of the poor’ 
\citep[p.~2]{pogge:2005_world}.


\subsection{Two perspectives on poverty-caused harms}

needs-based: We citizens of affluent countries
have a positive duty to meet needs.

harm-based: We have a negative duty not to harm.


\subsection{Libertarians}
‘Libertarianism is a family of views in political philosophy.
[...] Libertarians strongly value individual freedom and see this as justifying
strong protections for individual freedom.
[...] Libertarians usually see the kind of large-scale, coercive wealth
redistribution in which contemporary welfare states engage as involving
unjustified coercion’
\citep{vandervossen:2019_libertarianism}.


 
\section{Pogge’s Argument}
 
Key premise: a just institional order cannot ‘foreseeably reproduce avoidable human rights deficits on a massive scale’.

Observation: ‘If the rich countries scrapped their protectionist barriers against 
imports from poor countries, the populations of the latter would 
benefit greatly: hundreds of millions would escape unemployment, 
wage levels would rise substantially, and incoming export revenues 
would be higher by hundreds of billions of dollars each year.’


\begin{enumerate}

\item ‘Global institutional arrangements
are causally implicated in the reproduction of massive severe poverty.’
 
\item ‘Governments of [...] affluent countries bear primary responsibility for these
global institutional arrangements and can foresee their detrimental
effects.’
 
\item ‘there is a feasible institutional alternative under which such severe and extensive poverty would not persist’
 
\item ‘many citizens of [...] affluent countries bear responsibility
for the global institutional arrangements their governments have
negotiated in their names.’

\end{enumerate}

Conclusion: ‘the citizens and governments of the affluent countries, in collusion with the ruling elites of many poor countries, are harming the global poor by imposing an unjust institutional order upon them’ \citep[p.~59]{pogge:2005_world}.
	

\section{Objection}

Key premise: ‘We [could] hypothesize about the
distributive outcomes that would be
likely to arise under [a] fair
international order and then compare
these outcomes with the ones associated with the actual international
order. The gap between the two sets of
outcomes tells us the degree of
responsibility of the actual order for
the outcomes it is associated with’
\citep[p.~23]{patten:2005_world}.
 
Observation: ‘even in a fair international environment there is no guarantee that the
policies needed to fight poverty will be introduced domestically ... even fairly democratic countries, operating under an international set of
rules that have been shaped for their own advantage, can routinely fail to
enact policies designed to help their poorest and most marginalized
citizens’
\citep[pp.~23--4]{patten:2005_world}.
 

\subsection{Dilemma}
After reforming the international system, 
would the affluent have absolved themselves of complicity in
the fate of the poor?

If a proponent of Pogge’s view answers no, she faces standard objections to libertarianism.

If a proponent of Pogge’s view answers yes, she seems to abandoning a needs-based, rather than (as claimed) an exclusively harm-based, perspective.


 


\section{Conclusion}    
As far as we have seen, \\
Pogge is right that \\
from weak assumptions about duties not to harm \\
\hspace*{3mm} (assumptions so weak even a Libertarian would accept them) \\
it is possible to derive \\
a radical conclusion about redistribution.

But unless you hold that
‘property and other rights of the privileged should [...] be regarded as so absolute as to override a duty to perform easy rescues’, 
you cannot deny that some duty to act arises from the needs of those in extreme poverty.

\columnbreak

\ 
\columnbreak

\ 


\vfill

\footnotesize
\bibliography{$HOME/endnote/phd_biblio}

\end{multicols*}

\end{document}
